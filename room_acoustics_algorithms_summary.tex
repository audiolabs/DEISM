\documentclass{article}
\usepackage[utf8]{inputenc}
\usepackage{amsmath}
\usepackage{amsfonts}
\usepackage{amssymb}
\usepackage{geometry}
\usepackage{listings}
\usepackage{xcolor}

\geometry{margin=1in}

\title{Room Acoustics Analysis Algorithms Summary}
\author{DEISM Project}
\date{\today}

\begin{document}

\maketitle

\section{Overview}
This document summarizes the algorithms and mathematical functions implemented in the room acoustics analysis script. The script performs parameter estimation, frequency response analysis, and time-domain analysis for room acoustics applications.

\section{Core Algorithms}

\subsection{Parameter Estimation Algorithm}
The main algorithm \texttt{get\_params(V, S, c, datain)} calculates room acoustics parameters based on different input modes:

\subsubsection{Input Parameters}
\begin{itemize}
    \item $V$: Room volume (m³)
    \item $S$: Surface area (m²)
    \item $c$: Speed of sound (m/s)
    \item \texttt{datain}: Input data array containing mode and parameter values
\end{itemize}

\subsubsection{Mode 1: Impedance-based Calculation}
When mode = "imp", the algorithm calculates $T_{60}$ and $\alpha$ from impedance $\zeta$:

\begin{align}
b &= \frac{1}{\zeta} \\
\text{ratio}_1 &= \frac{1 + b}{1 - b} \\
\text{ratio}_2 &= \frac{b + 1}{b - 1} \\
d &= \ln(|\text{ratio}_1|^2) + 2\Re\left[b(2 - b\ln(|\text{ratio}_2|))\right] \\
T_{60} &= \frac{24\ln(10)V}{cSd} \\
\alpha &= \frac{8\Re(\zeta)}{|\zeta|^2}\left[1 + \frac{\Re(\zeta)^2 - \Im(\zeta)^2}{\Im(\zeta)|\zeta|^2}\arctan\left(\frac{\Im(\zeta)}{1 + \Re(\zeta)}\right) - \frac{\Re(\zeta)}{|\zeta|^2}\ln(1 + 2\Re(\zeta) + |\zeta|^2)\right]
\end{align}

\subsubsection{Mode 2: Absorption Coefficient-based Calculation}
When mode = "abs", the algorithm uses optimization to find impedance from absorption coefficient:

\begin{align}
\text{Objective function: } f(z_r) &= \left|\frac{\alpha_{\text{ref}} - \alpha_{\text{est}}(z_r)}{\alpha_{\text{ref}}}\right| \times 100
\end{align}

The optimization uses the Levenberg-Marquardt algorithm with multiple initial guesses and fallback to grid search.

\subsubsection{Mode 3: Reverberation Time-based Calculation}
When mode = "t60", the algorithm calculates impedance from reverberation time:

\begin{align}
\text{Objective function: } f(z_r) &= \left|\frac{T_{60,\text{ref}} - T_{60,\text{est}}(z_r)}{T_{60,\text{ref}}}\right| \times 100
\end{align}

\subsection{Absorption Coefficient Calculation}
The function \texttt{get\_imp\_abs(z, aran)} calculates the absorption coefficient from impedance:

\begin{align}
\alpha_{\text{est}} &= \int_0^{\pi/2} \frac{4z\cos(\theta)\sin(2\theta)}{z^2\cos^2(\theta) + 2z\cos(\theta) + 1} d\theta
\end{align}

This integral is computed using the trapezoidal rule with 200 integration points.

\subsection{Reverberation Time Calculation}
The function \texttt{get\_imp\_t60(z, ref, V, S, c)} calculates reverberation time from impedance:

\begin{align}
b &= \frac{1}{z} \\
d &= \ln\left(\left|\frac{1 + b}{1 - b}\right|^2\right) + 2\Re\left[b\left(2 - b\ln\left(\left|\frac{b + 1}{b - 1}\right|\right)\right)\right] \\
T_{60} &= \frac{24\ln(10)V}{cSd}
\end{align}

\subsection{Frequency Response Analysis}
The frequency response calculation uses modal analysis:

\begin{align}
\phi_s &= \cos\left(\frac{\pi x_s}{L_x}\right)\cos\left(\frac{\pi y_s}{L_y}\right)\cos\left(\frac{\pi z_s}{L_z}\right) \\
\phi_r &= \cos\left(\frac{\pi x_r}{L_x}\right)\cos\left(\frac{\pi y_r}{L_y}\right)\cos\left(\frac{\pi z_r}{L_z}\right) \\
\sigma &= \frac{2c}{4\zeta} + \frac{2c}{3\zeta} + \frac{2c}{2.5\zeta} \\
k_d &= c\sqrt{\left(\frac{1}{4}\right)^2 + \left(\frac{1}{3}\right)^2 + \left(\frac{1}{2.5}\right)^2} + j\frac{\sigma}{2c} \\
k &= \frac{\omega}{c} \\
P(\omega) &= \frac{j\omega Q\rho\phi_s\phi_r}{K(k_d^2 - k^2)}
\end{align}

where:
\begin{itemize}
    \item $\phi_s, \phi_r$: Source and receiver mode shapes
    \item $\sigma$: Modal damping
    \item $k_d$: Complex wavenumber
    \item $P(\omega)$: Frequency response
\end{itemize}

\subsection{Time Domain Analysis}
The time domain response is obtained through inverse FFT:

\begin{align}
p(t) &= \Re\left[\mathcal{F}^{-1}\{P(\omega)\}\right]
\end{align}

\subsection{Decibel Conversion}
The utility function \texttt{db(x)} converts magnitude to decibels:

\begin{align}
\text{db}(x) &= 20\log_{10}(|x|)
\end{align}

\section{Optimization Algorithms}

\subsection{Levenberg-Marquardt Algorithm}
The script uses \texttt{scipy.optimize.least\_squares} with the Levenberg-Marquardt method for parameter estimation. The algorithm minimizes the sum of squares of residuals:

\begin{align}
\min_{x} \sum_{i=1}^{m} f_i(x)^2
\end{align}

\subsection{Grid Search Fallback}
When the optimization fails, a grid search is performed:

\begin{align}
z_{\text{optimal}} = \arg\min_{z \in [1, 100]} f(z)
\end{align}

\section{Numerical Integration}

\subsection{Trapezoidal Rule}
The absorption coefficient calculation uses the trapezoidal rule for numerical integration:

\begin{align}
\int_a^b f(x)dx \approx \frac{b-a}{2n}\left[f(a) + 2\sum_{i=1}^{n-1}f(x_i) + f(b)\right]
\end{align}

\section{Error Handling and Robustness}

\subsection{Complex Logarithm Handling}
The algorithm handles complex logarithms by using absolute values to avoid complex results:

\begin{align}
\ln(z) \rightarrow \ln(|z|) \text{ for numerical stability}
\end{align}

\subsection{Fallback Values}
When calculations produce non-finite results, fallback values are used:
\begin{itemize}
    \item $d = 10^{-6}$ for invalid damping calculations
    \item Error penalty of $10^6$ for optimization failures
\end{itemize}

\section{Implementation Notes}

\begin{itemize}
    \item The script uses NumPy for numerical computations
    \item SciPy is used for optimization and integration
    \item Matplotlib is used for visualization
    \item Complex arithmetic is handled carefully to avoid numerical instabilities
    \item Multiple initial guesses are used for robust optimization
    \item Warning suppression is implemented for cleaner output
\end{itemize}

\section{Physical Parameters}

The analysis uses the following physical constants and parameters:
\begin{itemize}
    \item Speed of sound: $c = 343$ m/s
    \item Air density: $\rho = 1.2$ kg/m³
    \item Room volume: $V = 198.5$ m³
    \item Surface area: $S = 214.46$ m²
    \item Sampling frequency: $f_s = 44100$ Hz
\end{itemize}

\end{document}